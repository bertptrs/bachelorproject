\chapter{Experiments}

For these experiments, we used the Cage11 dataset from the The University of Florida Sparse Matrix Collection and repeatedly checked a certain route.% TODO: this probably requires a citation
This matrix represents a sparse, directed and weighted graph, so the algorithm could be applied. We compare multiple instantiations of the algorithm in terms of response timing.

As a platform, we used the DAS 4 and its MPI implementation. We run algorithm with 8 instances per physical node, which is default for the DAS 4. % TODO standardize das notation
The DAS scheduler is responsible for placing algorithms on nodes. To circumvent jitter in timings results, we repeat each experiment 10 times.

\section{Expectations}

Graph algorithms generally do not parallellize well % TODO: Citation needed
so we do not expect major speedups. Any speedups will be visible the most when we use 8 or less instances. This is because the overhead of MPI communication within one physical node is less than between different nodes. As such, a performance drop is to be expected when a second node is added.

\section{Results}

The first thing to notice is that there is very little speedup, if any, when the algorithm is run in parallel, as seen in \autoref{fig:speedup_cage11}.\footnote{Only the fastest implementation was considered for the speedup graph. Other implementations followed a similar pattern.} The timings are more or less the same until 8 workers, after which there is a slight slowdown. This can be explained by the increased latency when communicating between physical nodes.

The speed up graph does show a slight speedup when using two workers. 

\begin{figure}[t]
  \includegraphics[width=\textwidth]{graphs/graph_cage11_26345_36017}
  \caption{Relative speedup when running a specific route on the Cage11 dataset.}
  \label{fig:speedup_cage11}
\end{figure}

TODO: discussion on different implementations

\begin{figure}[t]
  \includegraphics[width=\textwidth]{graphs/graph_implementations}
  \caption{Comparison of implementations by execution time.}
  \label{fig:implmementations}
\end{figure}

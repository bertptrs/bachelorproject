\chapter{Conclusion}

In this thesis, we have looked at the max-flow min-cut problem and several algorithms that solve it. We have then taken the Push-Lift algorithm and implemented it in the \emph{forelem} framework. We have then derived several parallel implementations and tested them against various graphs.

In our experiments, we have seen that --- even though one implementation is significantly better than the other --- both implementations do not scale well when run in parallel, even when using larger numbers of workers. This can be attributed the poor locality of the algorithm, requiring a lot of communication to run.

As communication appears to be the bottleneck, it is also the first site to improve. When splitting work between workers, nodes have been divided by their node number. If a more local subdivision could be made, (for instance, by using graph clustering detection techniques), less communication between workers would be necessary, greatly reducing the overhead. Workers only need to communicate between each other when information changes on the boundary between them. If those boundaries were minimized, so would the needed communication.

The communication layer itself could be reworked to be more efficient. Currently, it each \emph{push} and \emph{lift} operation is transferred separately, incurring MPI overhead for every operation communicated. Combining the results and possibly sending them asynchronously could also improve performance.

Most importantly, more work can be done in automating the translation from \emph{forelem} to code. In this thesis, we made an approximation by hand, but an automated translation could create more equivalent implementations and try more variations. The tested implementations varied only in underlying data structure, but data distribution could also easily adapted. This would also allow the lessons learned here to be applied to other problems.

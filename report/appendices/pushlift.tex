\chapter{\textit{forelem} implementation for Push-Lift}

The following code listing show the forelem specification for Push-Lift used in the experiment of this thesis.

The global variables assumed to be present are:

\begin{itemize}
	\item $source$ the source node
	\item $sink$ the sink node
	\item $V, E$ the sets of tuples described above.
\end{itemize}

After the algorithm complettes, \texttt{flowOut} contains the maximum flow from \texttt{source} to \texttt{sink}.

\label{app:pushlift}
\begin{lstlisting}[mathescape,numbers=left,xleftmargin=2em]
// Initialization
flowOut = 0;

// Initialize each height as 0
forelem (i; i $\in$ pV) {
	V[i].h = 0;
}

V[source].h = $|V|$;

// Perform the initial push
forelem (i; i $\in$ pE.u[source]) {
	V[e[i].v].e += e[i].c;
	e[i].f = e.[i].c;
}

// While there still is an element with excess
whilelem (i $\in$ pV : V[i].e $> 0$) {
	if (V[i].u == sink) {
		flowOut += V[i].e;
		V[i].e = 0;
		continue;
	}

	// Determine the smallest pushable edge
	edge = null;
	minHeight = V[i].h;
	
	// Consider original edges
	forelem (j $\in$ pE.u[V[i].u]) {
		if (E[j].f < E[j].c && V[E[j].v].h < minHeight) {
			edge = j;
			minHeight = V[E[j].v].h;
		}
	}
	
	// Consider backtracking
	forelem (j in $\in$ pE.v[V[i].u]) {
		if (E[j].f > 0 && V[E[j].v].h < minHeight) {
			edge = j;
			minHeight = V[E[j].u].h;
		}
	}
	
	if (edge != null) {
		// Perform a push
		if (E[edge].u == V[i].u) {
			delta = min(V[i].e, E[edge].c - E[edge].f)
			E[edge].f += delta
		} else {
			delta = min(V[i].e, E[edge].f)
			E[edge].f -= delta
		}
		V[i].e -= delta;
	} else {
		// Need to lift, find a new height.
		newHeight = $\infty$;
		forelem (j in $\in$ pE.v[V[i].u]) {
			if (E[j].f > 0) {
				newHeight = min(newHeight, V[E[j].v].h)
			}
		}
		forelem (j in $\in$ pE.u[V[i].u]) {
			if (E[j].f < E[j].c) {
				newHeight = min(newHeight, V[E[j].u].h)
			}
		}
		
		V[i].h = newHeight + 1;
	}
}
 
\end{lstlisting}